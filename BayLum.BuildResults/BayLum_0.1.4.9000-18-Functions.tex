\begin{table}[ht]
\centering
\begin{tabular}{rllllllll}
  \hline
 & Name & Title & Description & Version & m.Date & m.Time & Author & Citation \\ 
  \hline
1 & Age\_Computation & Bayesian analysis for the OSL age estimation of one sample & This function computes the age (in ka) of a sample according to the model developed in Combes and Philippe (2017), based on an output of  Generate\_DataFile  or  Generate\_DataFile\_MG .  A sample, for which data is avalilable in several BIN files, can be analysed. &  &  &  & Claire Christophe, Sebastian Kreutzer, Anne Philippe, Guillaume Guérin$<$br /$>$ &  \\ 
  2 & Age\_OSLC14 & Bayesian analysis for age estimation of OSL measerments and C-14 ages of various samples & This function compute an age of OSL data of at least two samples and calibrate 14C ages of samples to get an age (in ka).  Age of OSL data are computed according to the model given in Combes and Philippe (2017). Single-grain or Multi-grain OSL measurements can be analysed simultaneouly (with output of  Generate\_DataFile  or  Generate\_DataFile\_MG  or both of them using  combine\_DataFiles ). Samples, for which data is avalilable in several BIN files, can be analysed.  For C14 data, the user can choose one of the following radiocarbon calibration curve: Northern or Sourthen Hemisphere or marine atmospheric. &  &  &  & Claire Christophe, Anne Philippe, Guillaume Guerin, Sebastian Kreutzer$<$br /$>$ &  \\ 
  3 & AgeC14\_Computation & Bayesian analysis for C-14 age estimations of various samples & This function calibrates the C-14 age of samples to get an age (in ka). The user can choose one of the following radiocarbon calibration curve: Northern or Sourthen Hemisphere or marine atmospheric. It must be the same curve for all samples. &  &  &  & Claire Christophe, Anne Philippe, Guillaume Guérin, Sebastian Kreutzer$<$br /$>$ &  \\ 
  4 & AgeS\_Computation & Bayesian analysis for OSL age estimation of various samples & This function computes the age (in ka) of at least two samples according to the model developed in Combes and Philippe (2017), based on outputs of  Generate\_DataFile  or  Generate\_DataFile\_MG  or both of them using  combine\_DataFiles .  Samples, for which data is avalilable in several BIN files, can be analysed.  Single-grain or Multi-grain OSL measurements can be analysed simultaneouly. &  &  &  & Claire Christophe, Anne Philippe, Guillaume Guérin, Sebastian Kreutzer$<$br /$>$ &  \\ 
  5 & AgeS & Output of  AgeS\_Computation  function for the samples: "GDB5" and "GDB3"$<$br /$>$ & Output of  AgeS\_Computation  function for the samples: "GDB5" and "GDB3", there is no stratigraphic relation neither systematic errors. &  &  &  &  &  \\ 
  6 & AtmosphericNorth\_CalC14 & $<$br /$>$ Atmospheric North data for calibration of 14C age$<$br /$>$ & As 14C years is not equal to calendar years because atmospheric 14C concentration varies through time. Hence, data in AtmosphericNorth\_CalC14 allows a calibration for mid-latitude Northern Hemisphere atmospher reservoir. &  &  &  &  &  \\ 
  7 & AtmosphericSouth\_CalC14 & $<$br /$>$ Atmospheric South data for calibration of 14C age  & As 14C years is not equal to calendar years because atmospheric 14C concentration varies through time. Hence, data in AtmosphericSouth\_CalC14 allows a calibration for mid-latitude Southern Hemisphere atmospher reservoir. &  &  &  &  &  \\ 
  8 & combine\_DataFiles & Old function Concat\_DataFile() & Combine objects generated by  Generate\_DataFile  and  Generate\_DataFile\_MG & 0.1.1
 &  &  & Sebastian Kreutzer, IRAMAT-CRP2A, UMR 5060, CNRS - Université Bordeaux Montaigne (France), adapting$<$br /$>$ the idea from the function 'Concat\_DataFile()' by Claire Christophe.$<$br /$>$ &  \\ 
  9 & create\_ThetaMatrix & Create Theta Matrix & Create the  $\backslash$Theta  matrix with the shared uncertainties that can used as input in, e.g.,  AgeS\_Computation  and  Age\_OSLC14  which is used for the covariance matrix  $\backslash$Sigma  (Combès \& Philippe, 2017) & 0.1.0
 &  &  & Sebastian Kreutzer, IRAMAT-CRP2A, UMR 5060, CNRS-Université Bordeaux Montaigne (France), based$<$br /$>$ on an 'MS Excel' sheet by Guillaume Guérin, IRAMAT-CRP2A, UMR 5060, CNRS-Université Bordeaux Montaigne (France)$<$br /$>$ &  \\ 
  10 & DATA\_C14 & C14 cal age estiamte and its error$<$br /$>$ & C14 cal age estiamtes and theirs error of samples S-EVA-26510, S-EVA-26506, S-EVA-26507, S-EVA-26508. &  &  &  &  &  \\ 
  11 & DATA1 & DATA of sample named GDB3$<$br /$>$ \%\%   \~{}\~{} data name/kind ... \~{}\~{} $<$br /$>$ & list of objects: LT, sLT, ITimes, dLab, ddot\_env, regDose, J,K,Nb\_measurement obtained using  Generate\_DataFile  function with single-grain OSL measurementsl of the sample GDB3. \%\%  \~{}\~{} A concise (1-5 lines) description of the dataset. \~{}\~{} &  &  &  &  &  \\ 
  12 & DATA2 & DATA on sample named GDB5$<$br /$>$ \%\%   \~{}\~{} data name/kind ... \~{}\~{} $<$br /$>$ & list of objects: LT, sLT, ITimes, dLab, ddot\_env, regDose, J,K,Nb\_measurement obtained using  Generate\_DataFile  function  with single-grain OSL measurementsl of the sample GDB5. \%\%  \~{}\~{} A concise (1-5 lines) description of the dataset. \~{}\~{} &  &  &  &  &  \\ 
  13 & DATA3 & $<$br /$>$ DATA of sample named FER1 & list of objects: LT, sLT, ITimes, dLab, ddot\_env, regDose, J,K,Nb\_measurement obtained using  Generate\_DataFile  function with multi-grain OSL measurementsl of the sample FER1. \%\%  \~{}\~{} A concise (1-5 lines) description of the dataset. \~{}\~{} &  &  &  &  &  \\ 
  14 & Generate\_DataFile\_MG & Generates, from one (or several) BIN file(s) of Multi-grain OSL measurements, a list of luminescence$<$br /$>$ data and information before statistical analysis & This function is used to generate, from the BIN file(s), a list of values of: &  &  &  & Claire Christophe, Sebastian Kreutzer, Anne Philippe, Guillaume Guérin$<$br /$>$ &  \\ 
  15 & Generate\_DataFile & Generates, from one (or several) BIN-file(s) of Single-grain OSL measurements,$<$br /$>$ a list of luminescence data and information before statistical analysis & This function is used to generate, from the BIN file(s), a list of values of: Single-grain  OSL intensities and associated uncertainties, regenerative doses, etc., which will be the input of the Bayesian models. To be easy-to-use, this function requires a rigorous organisation - all needed files should be arranged in one folder - of informations concerning each BIN file.   It is possible to process data for various samples simultaneously and to consider more than one BIN file per sample. &  &  &  & Claire Christophe, Sebastian Kreutzer, Anne Philippe, Guillaume Guerin$<$br /$>$ &  \\ 
  16 & LT\_RegenDose & Plots Lx/Tx as a function of the regenerative dose & This function plots Lx/Tx values as a function of regenerative dose, for every selected aliquot and for each sample. &  &  &  & Claire Christophe, Sebastian Kreutzer, Anne Philippe, Guillaume Guérin$<$br /$>$ &  \\ 
  17 & Marine\_CalC14 & $<$br /$>$ Marine data for calibration of 14C age  & As 14C years is not equal to calendar years because atmospheric 14C concentration varies through time. Hence, data in marine\_CalC14 allows a calibration for hypothetical "global" marine reservoir. &  &  &  &  &  \\ 
  18 & MCMCsample & MCMC sample from the posterior distribution of the dataset GDB5$<$br /$>$ \%\%   \~{}\~{} data name/kind ... \~{}\~{} $<$br /$>$ & MCMC samples from the posterior distribution of "A" for age, "D" for palaeodose and "sD" for dispersion of equivalent doses around "D", of the data set GDB5. \%\%  \~{}\~{} A concise (1-5 lines) description of the dataset. \~{}\~{} &  &  &  &  &  \\ 
  19 & Model\_Age & JAGS models use in  Age\_Computation $<$br /$>$ & A list of JAGS models use to a Bayesian analysis of OSL age of one sample. There are models for various growth curves and various distrubution to describe equivalent dose distribution around the palaeodose. &  &  &  &  &  \\ 
  20 & Model\_AgeC14 & JAGS models use in  AgeC14\_Computation $<$br /$>$ & A list of JAGS models use to a Bayesian analysis of C14 calibration age of various sample. Stratigraphic relations can be taken in count to calibrate C14 ages. This ages take into account that some data can be an outlier. &  &  &  &  &  \\ 
  21 & Model\_AgeS & JAGS models use in  AgeS\_Computation & A list of JAGS models use to a Bayesian analysis of OSL age of various samples. There are models for various growth curves and various distrubution to describe equivalent dose distribution around the palaeodose. &  &  &  &  &  \\ 
  22 & Model\_Palaeodose & $<$br /$>$ JAGS models use in  Palaeodose\_Computation $<$br /$>$ & A list of JAGS models use to a Bayesian analysis of OSL palaeodose of one or various samples. There are models for various growth curves and various distrubution to describe equivalent dose distribution around the palaeodose. &  &  &  &  &  \\ 
  23 & ModelC14 & $<$br /$>$ Likelihood of C14 samples for JAGS models use in  Age\_OSLC14 $<$br /$>$ & A list of models for C14 data to define likelyhood in JAGS models. &  &  &  &  &  \\ 
  24 & ModelOSL & $<$br /$>$ Likelihood of OSL samples for JAGS models use in  Age\_OSLC14 $<$br /$>$ & A list of models for OSL data to define likelyhood in JAGS models. &  &  &  &  &  \\ 
  25 & ModelPrior & $<$br /$>$ Prior for JAGS models use in  Age\_OSLC14 $<$br /$>$ & A list to define prior in JAGS models, taking acount OSL data and C14 data in stratigraphic constraint. The difficulty is in the fact that each cases is different. The youngest sample can be a C14 as well as a OSL sample. To resolve this problem we consider diferent cases thanks to this list. &  &  &  &  &  \\ 
  26 & Palaeodose\_Computation & Bayesian analysis for the palaeodose estimation of various samples & This function computes the palaeodose (in Gy) of one or various samples according to the model developed in Combes et al (2015), based on an output of  Generate\_DataFile  or  Generate\_DataFile\_MG  or both of them using  combine\_DataFiles .  Samples, for which data is avalilable in several BIN files, can be analysed.  Single-grain or Multi-grain OSL measurements can be analysed simultaneouly. &  &  &  & Claire Christophe, Sebastian Kreutzer, Anne Philippe, Guillaume Guérin$<$br /$>$ &  \\ 
  27 & plot\_Ages & Create age plot & Create age plot & 0.1.3
 &  &  & Sebastian Kreutzer, IRAMAT-CRP2A, UMR 5060, CNRS - Université Bordeaux Montaigne (France), based on code$<$br /$>$ written by Claire Christophe$<$br /$>$ &  \\ 
  28 & plot\_MCMC & Plot MCMC trajectories and posterior distributions & This function uses the output of  rjags::jags.model  to visualise the traces of the MCMC and the corresponding densities. In particular it displays the posterior distributions of the age, if it is calculated, palaeodose and the equivalent dose dispersion parameters of the sample. The function output is very similar to plot output produced with the 'coda' package, but tailored to meet the needs in the context of the 'BayLum' package. & 0.1.3
 &  &  & Sebastian Kreutzer, IRAMAT-CRP2A, UMR 5060, CNRS-Université Bordeaux Montaigne (France). This function$<$br /$>$ is a re-written version of the function 'MCMC\_plot()' by Claire Christophe$<$br /$>$ &  \\ 
  29 & plot\_Scatterplots & Display Scatter Plot Matrix of the Bayesian Age Results & Create a hexbin plot matrix ( hexbin::hexplom ) of age results returned by the bayesian age calculation. & 0.3.1
 &  &  & Sebastian Kreutzer, IRAMAT-CRP2A, UMR 5060, CNRS - Université Bordeaux Montaigne (France),$<$br /$>$ based on the function 'ScatterSamples()' by Claire Christophe, Anne Philippe, Guillaume Guérin$<$br /$>$ &  \\ 
  30 & SC\_Ordered & Create stratigraphically ordered sample matrix & Construct the stratigraphic matrix used in the functions  AgeS\_Computation  and  AgeC14\_Computation  for samples that are all ordered by increasing age. &  &  &  & Claire Christophe, Anne Philippe, Sebastian Kreutzer, Guillaume Guérin$<$br /$>$ &  \\ 
  31 & SCMatrix & Definition of the stratigraphic constraint matrix & This function helps to define the stratigraphic relation between samples, with questions. The output of this function can be used in function  AgeS\_Computation . &  &  &  & Claire Christophe, Anne Philippe, Guillaume Guerin$<$br /$>$ &  \\ 
   \hline
\end{tabular}
\end{table}

